\documentclass[xcolor=table]{beamer}

\usetheme[secheader,compress]{Madrid} %Primary theme

\usepackage{verbatim}
\usepackage{graphicx}

%% UTM Colors
\definecolor{UTMblue}{rgb}{0.043137, 0.137254, 0.254901}
\definecolor{UTMorange}{rgb}{1.0, 0.509803, 0}

\setbeamercolor{palette primary}{bg=UTMblue,fg=white}
\setbeamercolor{palette secondary}{bg=UTMblue,fg=white}
\setbeamercolor{palette tertiary}{bg=UTMblue,fg=white}
\setbeamercolor{palette quaternary}{bg=UTMblue,fg=white}
\setbeamercolor{structure}{fg=UTMblue} % itemize, enumerate, etc
\setbeamercolor{section in toc}{fg=UTMblue} % TOC sections
\setbeamercolor{title}{fg=UTMorange}

\setbeamercolor{subsection in head/foot}{bg=UTMorange,fg=white}

%%%%%%%%%%% BEGIN MACROS %%%%%%%%%%%%%%%%%%
% frameT: Frame with title
\newcommand{\frameT}[2]{\frame{\frametitle{#1} #2}}

% frameF: Fragile frame with title
\newcommand{\frameF}[2]{
  \begin{frame}[fragile]
    \frametitle{#1}
    #2
  \end{frame}
}

% frameTop: Frame aligned t the top
\newcommand{\frameTop}[2]{\frame[t]{\frametitle{#1} #2}}


\newcommand{\tab}{\hspace{1cm}}

\newcommand{\spaceor}{\hspace{5pt} \textbf{or} \hspace{5pt}}

%%%%%%%%%%% END MACROS %%%%%%%%%%%%%%%%%%%%



\begin{document}

\title{Kronos}

\author{Enrique Tejeda and Reily Stanford}
\institute{UT-Martin}
\date{\today}

%%%%%%%%%%% BEGIN TITLE %%%%%%%%%%%%%%%%%%
\frame{\titlepage}

 %\section{Outline}
%%%%%%%%%%%% END TITLE  %%%%%%%%%%%%%%%%%%


\section{Introduction}
\frameT{Motivation} {
  \bigskip
  \begin{itemize}
    \item Paranautical Activity - A roguelike FPS with a low-poly art style
      \bigskip
    \item Experimentation with VR technologies
      \bigskip
    \item Learning game design elements with Unreal Engine
  \end{itemize}
}

\frameT{Introduction} {
	\begin{itemize}
		\item Kronos is a VR FPS roguelike
		\item Created using Unreal Engine 5
		\item 4 different enemies types
			\begin{itemize}
				\item 2 bosses
			\end{itemize}
		\item 4 weapons
		\item 6 items
	\end{itemize}
}

\frameT{Story} {
	\begin{itemize}
		\item Someone has used a time machine as it was implicitly stated not to, by interacting with their previous self. This interaction has scrambled time via the butterfly effect.
		\item Our character, Kronos, is traveling through time to return the timeline to its original order.
		\item He has the ability to manipulate time by rewinding to a previous state. This proves beneficial when fighting the opponents he meets along the way.
	\end{itemize}
}

\frameT{Project Goals} {
  Described what you are trying to accomplish, including ``stretch'' goals.
}

\section{Sections--a useful organizational tool.}

\frameT{}{
}



\begin{frame}[fragile]
\frametitle{Family Tree Knowledge Base}
Facts:
\begin{verbatim}
Verbatim is a great way of enumerating code/algorithmic ideas.
\end{verbatim}
\end{frame}


\frameT{How to include images} {
  %% \includegraphics[width=.7\linewidth]{figures/image.pdf}
}


\begin{frame}[fragile]
  \frametitle{Social Network Graph}
  \begin{figure}[ht]
    \begin{minipage}[b]{0.53\linewidth}
      \centering
      Minipages are a great way to
    \end{minipage}
    \hspace{0.5cm}
    \begin{minipage}[b]{0.4\linewidth}
      \centering
      Line up side-by-side content.

    \end{minipage}
  \end{figure}
  
\end{frame}


\frameT{Results} {
  Describe any results of your work here.

  \bigskip

  Things that worked?

  \bigskip

  Things that didn't work?
}

\frameT{Conclusions} {
  Some bullet points here to wrap things up.
}

\frameT{Any Questions?} {
  
  \begin{center}
    Questions?
  \end{center}
  \begin{center}
    Comments?
  \end{center}

  \bigskip

  Further project/author information:
  \begin{center}
    \includegraphics[width=4cm]{figures/6AnLddq.png}
  \end{center}
}

%\frameF{fragile test} {
%}

%% \frameF{Prolog Family Tree} {
%% \begin{verbatim}
%% hello
%% \end{verbatim}



%% }

%Empty Page
%\frameT{Frame 1}{
%}  


\end{document}
