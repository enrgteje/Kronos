\documentclass[xcolor=table]{beamer}

\usetheme[secheader,compress]{Madrid} %Primary theme

\usepackage{verbatim}
\usepackage{graphicx}

\graphicspath{{figures/}}

%% UTM Colors
\definecolor{UTMblue}{rgb}{0.043137, 0.137254, 0.254901}
\definecolor{UTMorange}{rgb}{1.0, 0.509803, 0}

\setbeamercolor{palette primary}{bg=UTMblue,fg=white}
\setbeamercolor{palette secondary}{bg=UTMblue,fg=white}
\setbeamercolor{palette tertiary}{bg=UTMblue,fg=white}
\setbeamercolor{palette quaternary}{bg=UTMblue,fg=white}
\setbeamercolor{structure}{fg=UTMblue} % itemize, enumerate, etc
\setbeamercolor{section in toc}{fg=UTMblue} % TOC sections
\setbeamercolor{title}{fg=UTMorange}

\setbeamercolor{subsection in head/foot}{bg=UTMorange,fg=white}

%%%%%%%%%%% BEGIN MACROS %%%%%%%%%%%%%%%%%%
% frameT: Frame with title
\newcommand{\frameT}[2]{\frame{\frametitle{#1} #2}}

% frameF: Fragile frame with title
\newcommand{\frameF}[2]{
  \begin{frame}[fragile]
    \frametitle{#1}
    #2
  \end{frame}
}

% frameTop: Frame aligned t the top
\newcommand{\frameTop}[2]{\frame[t]{\frametitle{#1} #2}}


\newcommand{\tab}{\hspace{1cm}}

\newcommand{\spaceor}{\hspace{5pt} \textbf{or} \hspace{5pt}}

%%%%%%%%%%% END MACROS %%%%%%%%%%%%%%%%%%%%



\begin{document}

\title{Kronos}

\author{Enrique Tejeda and Reily Stanford}
\institute{UT-Martin}
\date{\today}

%%%%%%%%%%% BEGIN TITLE %%%%%%%%%%%%%%%%%%
\frame{\titlepage}

 %\section{Outline}
%%%%%%%%%%%% END TITLE  %%%%%%%%%%%%%%%%%%


\section{Introduction}

\frameT{Terms} {
	\begin{itemize}
		\item Roguelike - Procedurally generated gameplay
		\bigskip
		\item FPS - First person shooter
		\bigskip
		\item Low-poly - Low number of polygons
		\bigskip
		\item VR - Virtual Reality
	\end{itemize}
}

\frameT{Motivation} {
  \begin{itemize}
    \item Paranautical Activity - A roguelike FPS with a low-poly art style
    \medskip
	\begin{center}
		\includegraphics[scale=0.15]{paranautical_activity}
	\end{center}	    
      \smallskip
    \item Experimentation with VR technologies
      \medskip
    \item Learning game design elements with Unreal Engine
  \end{itemize}
}

\frameT{Story} {
	\begin{itemize}
		\item Someone has broken the rules of time
		\bigskip
		\item Timeline is scrambled, up to Kronos to repair it
		\bigskip
		\item Kronos manipulates time to assist in restoring time
		\bigskip
		\item Apparent that time is off
		\medskip
	\end{itemize}
	\begin{center}
		\includegraphics[scale=0.3]{timeMessedUp}
	\end{center}
}

\frameT{Technology} {
	\begin{itemize}
		\item Created using Unreal Engine 5
		\bigskip
		\item Programmed using Unreal's Blueprints
		\medskip
		\includegraphics[scale=0.22]{blueprint}
		\smallskip
		\item Procedural Dungeon Plugin by BenPyton
	\end{itemize}
}

\frameT{Procedural Generation} {
	\begin{itemize}
		\item The dungeon itself is built using levels
		\bigskip
		\item Plugin uses a depth first search algorithm
		\bigskip
		\item Plugin connects levels to each other dynamically
		\bigskip
		\item The levels are procedurally connected to provide a fluid start to finish dungeon
	\end{itemize}
}

\frameT{Procedural Generation Continued} {
	\begin{figure}
		\centering
		\includegraphics[scale=.09]{dungeon_room.png}
		\caption{Top-down view of a single level}
	\end{figure}
	\medskip
	\begin{figure}
		\centering
		\includegraphics[scale=.08]{dungeon.png}
		\caption{A dungeon created by the plugin}
	\end{figure}
}

\frameT{2D/3D Mashup} {
	\begin{figure}[ht]
		\begin{minipage}[b]{0.3\linewidth}
    	\centering
    			\includegraphics[scale=0.2]{2D.png}
    	\end{minipage}
    	\hspace{0.5cm}
    	\begin{minipage}[b]{0.5\linewidth}
      		\centering
      		\begin{itemize}
      			\item The world is crafted using 3D elements
				\bigskip
				\item The enemies and items are represented as 2D sprites to capture the feel of early game designs like Doom (1993)
      		\end{itemize}
    	\end{minipage}
  	\end{figure}
}

\section{Details}

\frameT{Gameplay} {
	\begin{itemize}
		\item Each level is a different time period
		\bigskip
		\item Kronos is sent to different eras to eradicate his flagged enemies
		\bigskip
		\item Enemies and bosses will reflect the era of the level
		\bigskip
		\item Kronos is able to bring weapons across levels
	\end{itemize}
}

\frameT{Demonstration} {
	\begin{center}
		\includegraphics[scale=0.25]{demo}
	\end{center}
}

\frameT{Trials and Tribulations} {
	\begin{figure}[ht]
		\begin{minipage}[b]{0.3\linewidth}
    	\centering
    			\includegraphics[scale=0.16]{sprite.png}
    	\end{minipage}
    	\hspace{0.5cm}
    	\begin{minipage}[b]{0.5\linewidth}
      		\centering
      		\begin{itemize}
      			\item Locomotion Movement Implementation
				\bigskip
				\item Sprite Changing with the Different Actions
				\bigskip
				\item Understanding the Dungeon Creation Plugin with the Documentation Provided
      		\end{itemize}
    	\end{minipage}
  	\end{figure}
}

\section{Conclusion}

\frameT{Future Work} {
	\begin{itemize}
		\item Create a new floor (possibly a modern era)
		\bigskip
		\item AI creation and management using Behavior Trees
		\bigskip
		\item More enemies, bosses, and weapons
	\end{itemize}
}

\frameT{Feedback} {
	\begin{itemize}
		\item Any questions?
		\bigskip
		\item Project repo: https://github.com/enrgteje/Kronos
	\end{itemize}
	\bigskip
	\bigskip
	\centering
	Contact us:
		\begin{figure}[ht]
    		\begin{minipage}[b]{0.4\linewidth}
      			\centering
      			enrgteje@ut.utm.edu
      			github.com/enrgteje
    		\end{minipage}
    		\hspace{0.5cm}
    		\begin{minipage}[b]{0.4\linewidth}
      			\centering
      			rstanfo1@ut.utm.edu
      			github.com/reilys
    		\end{minipage}
  		\end{figure}
	
}


%\frameT{Project Goals} {
%  Described what you are trying to accomplish, including ``stretch'' goals.
%}

%\section{Sections--a useful organizational tool.}
%
%\frameT{}{
%}
%
%
%
%\begin{frame}[fragile]
%\frametitle{Family Tree Knowledge Base}
%Facts:
%\begin{verbatim}
%Verbatim is a great way of enumerating code/algorithmic ideas.
%\end{verbatim}
%\end{frame}
%
%
%\frameT{How to include images} {
%  %% \includegraphics[width=.7\linewidth]{figures/image.pdf}
%}
%
%
%\begin{frame}[fragile]
%  \frametitle{Social Network Graph}
%  \begin{figure}[ht]
%    \begin{minipage}[b]{0.53\linewidth}
%      \centering
%      Minipages are a great way to
%    \end{minipage}
%    \hspace{0.5cm}
%    \begin{minipage}[b]{0.4\linewidth}
%      \centering
%      Line up side-by-side content.
%
%    \end{minipage}
%  \end{figure}
%  
%\end{frame}
%
%
%\frameT{Results} {
%  Describe any results of your work here.
%
%  \bigskip
%
%  Things that worked?
%
%  \bigskip
%
%  Things that didn't work?
%}
%
%\frameT{Conclusions} {
%  Some bullet points here to wrap things up.
%}
%
%\frameT{Any Questions?} {
%  
%  \begin{center}
%    Questions?
%  \end{center}
%  \begin{center}
%    Comments?
%  \end{center}
%
%  \bigskip
%
%  Further project/author information:
%  \begin{center}
%    \includegraphics[width=4cm]{6AnLddq.png}
%  \end{center}
%}

%\frameF{fragile test} {
%}

%% \frameF{Prolog Family Tree} {
%% \begin{verbatim}
%% hello
%% \end{verbatim}



%% }

%Empty Page
%\frameT{Frame 1}{
%}  


\end{document}
